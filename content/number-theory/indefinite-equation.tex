
\section{不定方程}
\label{sec:indefinite-equation}

\subsection{二元一次不定方程}
\label{sec:indefinite-binary-equation-of-the-first-degree}

\begin{theorem}
  二元一次不定方程$ax+by=c$有整数解的充分必要条件是$a$和$b$的最大公因数$d$能够整除$c$.
\end{theorem}

\begin{proof}[证明]
  先证必要证,如果这方程有解,比如说$x=x_0,y=y_0$是它的一个解,则$ax_0+by_0=c$,显然$a$和$b$的最大公因数$d$能够整除左边,自然也就能整除$c$,必要性成立。

  再证充分性,如果$d=(a,b) \mid c$,记$c=c_1d$,由最大公因数性质,存在整数$s$和$t$使得$sa+tb=d$,于是$(c_1s)a+(c_1t)b=c$,从而$x=c_1s,y=c_1t$是方程的一个解,充分性成立。
\end{proof}

\begin{theorem}
  如果二元一次不定方程$ax+by=c$有一个解$x=x_0,y=y_0$,则它的全部解是
  \[ x=x_0+b_1t, \  y=y_0-a_1t \  (t=0, \pm 1, \pm 2, \ldots ) \]
  其中$a_1$和$b_1$分别是从$a$和$b$中约去它们的最大公因数后剩下的因子。
\end{theorem}

\begin{proof}[证明]
  显然$x=x_0+b_1t,y=y_0-a_1t,(t=0, \pm 1, \pm 2, \ldots)$都是这方程的解,只要证明它给出了方程的全部解即可,设$x=x_1,y=y_1$是方程的任意一个解,即$ax_1+by_1=c$,则
  \[ a(x_1-x_0)+b(y_1-y_0)=0 \]
  两边约去$a$和$b$的最大公因数$d$后得
  \[ a_1(x_1-x_0)+b_1(y_1-y_0)=0 \]
  这表明$a_1 \mid b_1(y_1-y_0)$且$b_1 \mid a_1(x_1-x_0)$,但$(a_1,b_1)=1$,所以必有$a_1 \mid (y_1-y_0)$, $b_1 \mid (x_1-x_0)$,记$y_1-y_0=a_1t_1$,$x_1-x_0=b_1t_2$,则由上式得$t_1+t_2=0$,所以记$t=t_2$,则$x_1=x_0+b_1t,y_1=y_0-a_1t$,这就表明定理中的解给出了方程的全部解。
\end{proof}

\subsection{勾股方程与费马问题}
\label{sec:pythagorean-equation-and-fermat-problem}


%%% Local Variables:
%%% mode: latex
%%% TeX-master: "../../elementary-math-note"
%%% End:
