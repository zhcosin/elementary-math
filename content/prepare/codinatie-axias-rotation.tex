
\section{坐标变换}
\label{sec:codinatie-axias-rotation}

在高级中学数学教材中,三角函数的伸缩变换已经为人所熟知,而本文要讨论的是坐标变换的一般性理论。

\subsection{平移}

平移变换由平移向量$\vv{s}=(a,b)$唯一确定,平面上任意一点$P$与它在变换下的像$P'$满足$\vv{PP'}=\vv{s}$为常向量,因此如果点$P$的坐标是$(x,y)$,点$P'(x',y')$的坐标将是$(x+a, y+b)$,即
\begin{equation}
  \label{eq:shift-translation-coordinate-formula}
  \left\{
      \begin{array}{ccc}
        x' & = & x + a \\
        y' & = & y + b
      \end{array}
  \right.
\end{equation}
写成矩阵形式则是:
\begin{equation}
  \label{eq:shift-translation-coordinate-formula-matrix}
  \left(
    \begin{array}{c}
      x' \\
      y' \\
      1
    \end{array}
  \right)
  =
  \left(
    \begin{array}{ccc}
      1 & 0 & a \\
      0 & 1 & b \\
      0 & 0 & 1
    \end{array}
  \right)
  \left(
    \begin{array}{c}
      x \\
      y \\
      1
    \end{array}
  \right)
\end{equation}

\begin{example}
  本目中来建立曲线$f(x,y)=0$在平移向量为$\vv{s}=(a,b)$的平移变换下的新方程,假使点$P(x,y)$为新曲线上任一点,则它在变换前的坐标则是$(x-a,y-b)$,而变换前的点是满足原曲线方程的,所以得到新坐标所满足的方程$f(x-a,y-b)=0$,此即原曲线在平移变换下的新的曲线的方程。比如说,以原点为圆心的圆$x^2+y^2=r^2$在此平移变换下的新方程即为$(x-a)^2+(y-b)^2=r^2$。对于一般函数$y=g(x)$而言,把它写成$f(x,y)=g(x)-y$即可应用此结论。
\end{example}

\subsection{旋转}
假如在平面直角坐标系中,有一个点的坐标是$P(x,y)$,现在我们把它绕着原点逆时针方向转动一个角度$\theta$,我们来寻求这个点的新坐标$P'=(x', y')$与原坐标之间的关系。

设向量$\overrightarrow{OP}$与坐标系$x$轴正向成角$\alpha$,则向量$\vv{OP'}$与坐标系$x$轴正向成角$\alpha+\theta$,记向量$\overrightarrow{OP}$长度为$r$,则
\begin{equation*}
  x=r\cos{\alpha},y=r\sin{\alpha}
\end{equation*}
同样有
\begin{equation*}
  x'=r\cos{(\alpha+\theta)},y'=r\sin{(\alpha+\theta)}
\end{equation*}
于是就有
\begin{equation}
  \label{eq:formulas-rotation-axias}
  \begin{split}
  x' & = x\cos{\theta} - y\sin{\theta} \\
  y' & = x\sin{\theta} + y\cos{\theta}
  \end{split}
\end{equation}
写成矩阵形式就是:
\begin{equation*}
  \left(
    \begin{array}{c}
      x' \\
      y'
    \end{array}
  \right)
    =
    \left(
      \begin{array}{cc}
        \cos{\theta} & -\sin{\theta} \\
        \sin{\theta} & \cos{\theta}
      \end{array}
    \right)
  \left(
    \begin{array}{c}
      x \\
      y
    \end{array}
  \right)
\end{equation*}
或者写成这样:
\begin{equation*}
  \left(
    \begin{array}{c}
      x' \\
      y' \\
      1
    \end{array}
  \right)
    =
    \left(
      \begin{array}{ccc}
        \cos{\theta} & -\sin{\theta} & 0 \\
        \sin{\theta} & \cos{\theta} & 0 \\
        0 & 0 & 1
      \end{array}
    \right)
  \left(
    \begin{array}{c}
      x \\
      y \\
      1
    \end{array}
  \right)
\end{equation*}

\begin{example}
  反比例函数$xy=1$的图象位于一三象限并且关于这两个象限的角平分线对称,我们尝试把它的图象绕原点顺时针旋转45度,看看新的曲线方程的模样。为此目的,记反比例曲线为$C$,顺时针旋转45度后得到的新曲线记为$C'$,我们反过来把$C$看成是由$C'$绕原点逆时针旋转45度得到的,就有关系式$x=\frac{\sqrt{2}}{2}(x'-y'), y=\frac{\sqrt{2}}{2}(x'+y')$,因此新曲线$C'$上的点$P(x',y')$都满足方程$\frac{\sqrt{2}}{2}(x'-y') \cdot \frac{\sqrt{2}}{2}(x'+y')=1$,即$x'^2-y'^2=2$,这是一个离心率为$\sqrt{2}$的双曲线,所以反比例函数的图象是双曲线。
\end{example}

\subsection{伸缩}


%%% Local Variables:
%%% mode: latex
%%% TeX-master: "../../book"
%%% End:
