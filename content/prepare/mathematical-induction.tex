
\section{数学归纳法}
\label{sec:mathematical-induction}
除了常用的第一数学归纳法以外,我们还有以下的:
\begin{principle}[第二数学归纳法]
如果与正整数有关的命题$P(n)$满足:
  \begin{enumerate}
  \item $P(1)$成立;
  \item 由$P(1),P(2),\dots,P(k)$成立能够推证出$P(k+1)$成立;
  \end{enumerate}
那么该命题对于一切正整数成立.
\end{principle}
有些命题在递推过程中,依赖的是前面的所有结论而非仅仅依赖前一个结论,此时第二数学归纳法就非常适用。
\begin{principle}[倒推归纳法]
如果与正整数有关的命题$P(n)$满足:
  \begin{enumerate}
  \item 有无穷多个正整数$n$使命题成立;
  \item 由$P(k)$成立能够推证$P(k-1)$成立;
  \end{enumerate}
那么该命题对于一切正整数成立。
\end{principle}
倒推归纳法的原理也是显而易见的,对于任何一个给定的正整数$n$,由于不超过$n$的正整数只有有限个,所以必然存在一个大于$n$的正整数$N$,使得命题$P(N)$成立,再倒推回来,知$P(n)$成立。

\begin{example}[均值不等式]
作为一个例子,我们用倒推归纳法来证明均值不等式,这比用通常的第一数学归纳法来得更加容易:

  均值不等式的内容是: 对任意$n(n\geqslant2)$个正实数$a_i$,有下面不等式成立:
\[ \frac{1}{n}\sum_{i=1}^na_i \geqslant \left( \prod_{i=1}^na_i \right)^{\frac{1}{n}} \]

\begin{proof}[证明]\footnote{这个倒推归纳法的证明来自于参考文献\cite{the-secret-of-inequality}.}
  对$n=2$的情形,有$\frac{1}{2}(a_1+a_2)-\sqrt{a_1a_2}=\frac{1}{2}(\sqrt{a_1}-\sqrt{a_2})^2\geqslant 0$知不等式成立。

 反复使用$n=2$的结论,我们就可以得到当$n$是2的幂的时候不等式是成立的,然而2的幂是无穷多的,所以只要证明,不等式如果对$n+1$个正实数成立就必然对$n$个正实数也成立就可以了。

对于任意$n$个正实数,我们再添加一个正实数$a_{n+1}=\frac{1}{n}\sum_{i=1}^na_i$构成$n+1$个正实数,由假设,不等式对$n+1$个正实数是成立的,所以有
\[
\frac{1}{n+1}\sum_{i=1}^{n+1}a_i \geqslant \left( \prod_{i=1}^{n+1}a_i \right)^{\frac{1}{n+1}}
\]
而由于$a_{n+1}$正好等于其它$n$个实数的平均数,所以
\[ \frac{1}{n+1}\sum_{i=1}^{n+1}a_i = \frac{1}{n}\sum_{i=1}^{n}a_i \]
因此前一式即为:
\[
\frac{1}{n}\sum_{i=1}^{n}a_i \geqslant \left( \prod_{i=1}^{n}a_i \right)^{\frac{1}{n+1}} \cdot \left( \frac{1}{n}\sum_{i=1}^na_i \right)^{\frac{1}{n+1}}
\]
化简即得
\[
\frac{1}{n}\sum_{i=1}^na_i \geqslant \left( \prod_{i=1}^na_i \right)^{\frac{1}{n}} 
\]
即得证。
\end{proof}
\end{example}

\begin{principle}[跳跃数学归纳法]
  如果与正整数有关的命题$P(n)$满足:
  \begin{enumerate}
  \item $P(1)$,$P(2)$,$\ldots$,$P(m)$成立;
  \item 由$P(k)$成立能够推证$P(k+m)$成立;
  \end{enumerate}
那么该命题对于一切正整数成立。
\end{principle}

数学归纳法还有其他形式,比如解决对偶性问题(如正余弦)的螺旋归纳法,解决同时与两个正整数相关的命题的二重归纳法等,但它们的道理都是类似的。

数学归纳法的奠基也并非一定要从1开始。

%%% Local Variables:
%%% mode: latex
%%% TeX-master: "../../elementary-math-note"
%%% End:
