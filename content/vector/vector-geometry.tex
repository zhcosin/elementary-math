
\section{用向量定比分点公式解决与交点相关的平面几何问题}
\label{sec:vector-geometry}

本文是人教数学官方论坛QQ群里某网友发上来的题目,我一瞅,这不又是这类问题么,之前 我用向量定比分点公式证明过梅涅劳斯定理和塞瓦定理,少不得再卖弄一番。当然了,其实葫芦里 也没什么东西,不过是用了向量的共线、定比分点和内积方面的几个小知识点而已,借此也保证不 超出高中教学大纲的范围。

向量的定比分点公式,其实更加有用的是另一种形式$\vec{x} = t \vec{a} + (1 - t) \vec{b}$,因为只要取$t = 1 / 2$便 定位出中点,而分别取0和1,则分别得出两个基向量,这是用分比的形式所得不出的。

这个方法就是,对于两条线段的交点,先在其中一条线段上使用定比分点公式,而公式中的这 个实数就是一个待定的参数,怎么定出这个参数呢,再通过该点在另一线段上,就有相关向量的共 线关系,于是得出关于这个参数的方程,以后的事情就没什么技术含量了。

题目: 三角形$\triangle A B C$中,$A D$是$\angle B A C$的平分线,点$D$在$B C$边上,$D E$、$D F$分别与边$A B$、$A C$垂直, $E$和$F$是垂足,$B F$、$C E$相交于点$P$,求证:$A P \bot B C$。

证明:设$\overrightarrow{A B} = \vec{a}$, $\overrightarrow{A C} = \vec{b}$,并记$| \vec{a} | = a$,$| \vec{b} | = b$,则由于$\frac{B D}{D C} = \frac{A B}{A C}$,根据向量定比分点公式得 
\begin{equation}
\overrightarrow{A D} = \frac{b}{a + b} \vec{a} + \frac{a}{a + b} \vec{b}
\end{equation}
 再设$\overrightarrow{A E} = \lambda_1 \vec{a}$,则由$\overrightarrow{D E} \cdot \vec{a} = 0$得
 \begin{equation}
   \label{eq:vector-geometry-lambda-1}
(b - \lambda_1 (a + b)) a + (\vec{a} \cdot \vec{b}) = 0 
 \end{equation}
或者
\begin{equation}
  \label{eq:vector-geometry-lambda-1-1}
(1 - \lambda_1) a b - \lambda_1 a^2 + m = 0
\end{equation}

同理,设$\overrightarrow{A F} = \lambda_2 \vec{b}$,则有
\begin{equation}
  \label{eq:vector-geometry-lambda-2}
(a - \lambda_2 (a + b)) b + (\vec{a} \cdot \vec{b}) = 0
\end{equation}
或者
\begin{equation}
  \label{eq:vector-geometry-lambda-2-1}
(1 - \lambda_2) a b - \lambda_2 b^2 + m = 0
\end{equation}
(题外话:据此便有$a \lambda_1 = b \lambda_2$,这与$A E = A F$相吻合)

为确定点$P$的位置,设$\overrightarrow{B P} = t \overrightarrow{P F}$,则
\begin{align}
\overrightarrow{A P} & = & \frac{1}{1 + t} \overrightarrow{A B} + \frac{t}{1 + t} \overrightarrow{A F} \\
& = & \frac{1}{1 + t} \vec{a} + \frac{t}{1 + t} \lambda_2 \vec{b}
\end{align}

为便于计算,记$r = \frac{t}{1 + t}$,则$\overrightarrow{A P} = (1 - r) \vec{a} + r \lambda_2 \vec{b}$ 

接着便要确定$r$的值,由于点$P$也在$C E$上,所以$\overrightarrow{C P}$与$\overrightarrow{C E}$共线,而 $\overrightarrow{C E} = \overrightarrow{A E} - \overrightarrow{A C} = \lambda_1 \vec{a} - \vec{b}$,$\overrightarrow{C P} = \overrightarrow{A P} - \overrightarrow{A C} = (1 - r) \vec{a} + (r \lambda_2 - 1) \vec{b}$

设$\overrightarrow{C P} = s \overrightarrow{C E}$,消去$s$得到 $1 - r = \lambda_1 (1 - r \lambda_2)$,记为(3)式,解之得$r = \frac{1 - \lambda_1}{1 - \lambda_1 \lambda_2}$ ,记为(3a)式

同样为了方便,记$m = (\vec{a} \cdot \vec{b})$,则
\begin{eqnarray*}
  \overrightarrow{C B} \cdot \overrightarrow{A P} & = & (\vec{a} - \vec{b})
  \cdot ((1 - r) \vec{a} + r \lambda_2 \vec{b})\\
  & = & (1 - r) a^2 - r \lambda_2 b^2 + (r \lambda_2 - (1 - r)) m
\end{eqnarray*}
由(2a)有 $\lambda_2 b^2 = (1 - \lambda_2) a b + m$,代入上式得
$\begin{array}{lll}
  \overrightarrow{C B} \cdot \overrightarrow{A P} & = & (1 - r) a^2 - r (1 -
  \lambda_2) a b - r m + (r \lambda_2 -  1 + r)) m\\
  & = & (1 - r) a^2 - r (1 - \lambda_2) a b + (r \lambda_2 -  1) m
  
\end{array}$
又由(3)式得$r \lambda_2 - 1 = - \frac{1 -
r}{\lambda_1}$,并将(3a)代入上式得
$\begin{array}{lll}
  \overrightarrow{C B} \cdot \overrightarrow{A P} & = & (1 - r) \left( a^2 -
  \frac{m}{\lambda_1} \right) - \frac{(1 - \lambda_1) (1 - \lambda_2)}{1 -
  \lambda_1 \lambda_2} a b\\
  & = & \frac{\lambda_1 (1 - \lambda_2)}{1 - \lambda_1 \lambda_2} \left( a^2
  - \frac{m}{\lambda_1} \right) - \frac{(1 - \lambda_1) (1 - \lambda_2)}{1 -
  \lambda_1 \lambda_2} a b\\
  & = & \frac{1 - \lambda_2}{1 - \lambda_1 \lambda_2} (\lambda_1 a^2 - m - (1
  - \lambda_1) a b)\\
  & = & \frac{1 - \lambda_2}{1 - \lambda_1 \lambda_2} a (\lambda_1 a (a + b)
  - m - a b)
\end{array}$
再由(1)式得$\lambda_1 a (a + b) = a b + m$,代入上式立刻得到
$\overrightarrow{C B} \cdot \overrightarrow{A P} = 0$,证毕。





