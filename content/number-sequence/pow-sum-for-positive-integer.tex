
\section{正整数的幂和公式}
\label{sec:power-sum-for-positive-integer}

本文来计算正整数的幂和,也就是求$S_m(n) = \sum_{k=1}^nk^m$的关于$n$的公式。

显然有$s_0(n)=n$,然后利用二项式展开有
\[ (k+1)^{m}-k^{m} = \sum_{i=0}^{m-1}C_{m}^{i}k^i \]
令$k=1,2,\cdots,n$进行累加即得
\begin{eqnarray}
  \label{eq:pow-sum-positive-integer-recursive}
(n+1)^{m}-1 & = & \sum_{i=0}^{m-1}C_{m}^{i}\sum_{k=1}^{n}k^i \\
             & = & \sum_{i=0}^{m-1}C_{m}^{i}S_i(n)
\end{eqnarray}
因此,在上式中取$m=2$得
\[ (n+1)^2-1 = S_0(n)+2S_1(n) \]
于是(结果已经进行因式分解)
\[ S_1(n)=\frac{1}{2}n(n+1) \]
继续取$m=3$得
\[ (n+1)^3-1=S_0(n)+3S_1(n)+3S_2(n) \]
并将$S_0(n)$和$S_1(n)$代入即得
\[ S_2(n) = \frac{1}{6}n(n+1)(2n+1) \]
同样可以继续得到
\[ S_3(n)=\frac{1}{4}n^2(n+1)^2 \]
继续下去,理论上可以得到正整数的任意正整数次方的和式。

从\ref{eq:pow-sum-positive-integer-recursive}还可以得出:$S_m(n)$是关于$n$的$m+1$次多项式。