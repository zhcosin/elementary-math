
\section{关于正整数倒数和的一些讨论}
\label{sec:sum-reciprocal-of-positive-integer}

本节讨论与正整数倒数和相关的一些问题。

首先是 $S_n=\sum_{k=1}^n\frac{1}{k}$,这是一个递增的数列,但是增量逐次减小趋于零,所以一个很自然的问题就是:这数列是有一个上限呢,还是可以无限增加到正无穷。

结论是它在增加过程中可以超过任何一个正实数,无论多大的正实数,也就是说,它是一个正无穷大。

先叙述一个在许多书上都可以找到的证明,这个证明是这样的,把正整数序列进行分段,1 为第零段,2为第一段,3和4为第二段,5到8为第三段,一般的第$n$段是从$2^{n-1}+1$到$2^n$,即每一段的最后一个数都刚好是2的幂,于是对于每一段上的正整数们的倒数,显然最后一个正整数的倒数是最小的,所以该段上的正整数的倒数和大于$2^{n-1}\cdot \frac{1}{2^n}=\frac{1}{2}$,而按照这个方法,这个段是可以有无穷多的,只是越往后,段的长度就越长,所以这就说明它了$S_n$是无限增加的。

再提一个思路完全一样只是分段方式不同的证明:由于函数$y=\frac{1}{x}$在正实数区间上是下凸函数,对于任何两个正整数$m$和$n(>m)$,根据琴生不等式有:
\begin{equation*}
  \frac{1}{n-m} + \frac{1}{n+m} > \frac{2}{n}
\end{equation*}
于是按照首尾配对相加的方式可得如下不等式(其实可以使用琴生不等式直接得出)
\begin{equation*}
  \frac{1}{k+1} + \cdots + \frac{1}{k+m+1} + \cdots + \frac{1}{k+2m+1} > \frac{2m+1}{k+m+1}
\end{equation*}
取$m=k$,就有
\begin{equation*}
  \frac{1}{k+1} + \cdots + \frac{1}{3k+1} > 1
\end{equation*}
如此1为第零段,2到4为第一段,5到13为第三段,如此下去,也能达到同样的目标。

接下来考虑$I_n=1-\frac{1}{2}+\frac{1}{3}-\frac{1}{4}+\cdots+\frac{1}{2n-1}-\frac{1}{2n}$,我们来尝试使用简单表达式逼近它。

记$J_n=\sum_{k=1}^n(\frac{1}{2k}-\frac{1}{2k+1})$,由于$z_n=\frac{1}{n-1}-\frac{1}{n}=\frac{1}{n(n+1)}$是递减的,并且相邻两项也相差越来越小,所以有不等式$z_{2k}<\frac{1}{2}(z_{2k-1}+z_{2k})$,也就是如下的:
\begin{equation*}
  \frac{1}{2k-1} - \frac{1}{2k} < \frac{1}{2} \left[ \left( \frac{1}{2k-2} - \frac{1}{2k-1} \right) + \left( \frac{1}{2k} - \frac{1}{2k+1} \right) \right]
\end{equation*}
对上式左边进行累加,但从$k=3$到$k=n$使用右边放缩,得
\begin{equation*}
  I_n<\frac{1}{2}+\frac{1}{12}+\frac{1}{2} \left[ \left( J_n-\frac{1}{6}-\frac{1}{2n(2n+1} \right) + \left( J_n-\frac{1}{6}-\frac{1}{20} \right) \right]
\end{equation*}
化简
\begin{equation*}
 I_n<J_n+\frac{47}{120} - \frac{1}{4n(2n+1)}
\end{equation*}
利用$I_n+J_n=1-\frac{1}{2n+1}$从上式中换掉$J_n$得
\begin{equation}
  \label{eq:sign-sum-reciprocal-positive-integer-max}
  I_n<\frac{167}{240}-\frac{1}{2(2n+1)}-\frac{1}{8n(2n+1)}
\end{equation}
这就是一个上限估计,再来考虑下限,同样因为$z_n$是递减的,有不等式$z_{2k}>\frac{1}{2}(z_{2k}+z_{2k+1})$,也就是
\begin{equation*}
  \frac{1}{2k-1} - \frac{1}{2k} > \frac{1}{2} \left[ \left( \frac{1}{2k-1} - \frac{1}{2k} \right) + \left( \frac{1}{2k} - \frac{1}{2k+1} \right) \right]
\end{equation*}
对左边进行累加,在$k \geqslant 3$ 时使用右边放缩,得到
\begin{equation*}
  I_n > \frac{1}{2} + \frac{1}{12} + \frac{1}{2} \left( \frac{1}{5} - \frac{1}{2n+1} \right)
\end{equation*}
也就是
\begin{equation}
  \label{eq:sign-sum-reciprocal-positive-integer-min}
 I_n > \frac{41}{60} -\frac{1}{2(2n+1)} 
\end{equation}
式 \ref{eq:sign-sum-reciprocal-positive-integer-max} 和 \ref{eq:sign-sum-reciprocal-positive-integer-min}即构成$I_n$的一个估计,这个逼近程度怎么样呢,取$n \geqslant 100$,得
\begin{equation*}
  0.680845771... < I_n < 0.695833333...
\end{equation*}
说明这个逼近程度还是比较理想的。

%%% Local Variables:
%%% mode: latex
%%% TeX-master: "../../book"
%%% End: