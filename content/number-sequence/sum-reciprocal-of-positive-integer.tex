
\section{关于正整数倒数和的一些讨论}
\label{sec:sum-reciprocal-of-positive-integer}

本节讨论与正整数倒数和相关的一些问题。

首先是 $S_n=\sum_{k=1}^n\frac{1}{k}$,这是一个递增的数列,但是增量逐次减小趋于零,所以一个很自然的问题就是:这数列是有一个上限呢,还是可以无限增加到正无穷。

结论是它在增加过程中可以超过任何一个正实数,无论多大的正实数,也就是说,它是一个正无穷大。

先叙述一个在许多书上都可以找到的证明,这个证明是这样的,把正整数序列进行分段,1 为第零段,2为第一段,3和4为第二段,5到8为第三段,一般的第$n$段是从$2^{n-1}+1$到$2^n$,即每一段的最后一个数都刚好是2的幂,于是对于每一段上的正整数们的倒数,显然最后一个正整数的倒数是最小的,所以该段上的正整数的倒数和大于$2^{n-1}\cdot \frac{1}{2^n}=\frac{1}{2}$,而按照这个方法,这个段是可以有无穷多的,只是越往后,段的长度就越长,所以这就说明它了$S_n$是无限增加的。

再提一个思路完全一样只是分段不同的证明,可以证明,对于正整数$n>3$,下面不等式成立:
\begin{equation*}
  \frac{1}{n+1} + \frac{1}{n+2} \cdots + \frac{1}{3n} > 1
\end{equation*}
实际上,左边是一个递增的(作差即知),而在$n=4$时左边已经大于1,所以得出。

如此1、2、3、4各自独立为一段,以后5到12为一段,13到36为一段,如此下去,也能达到同样的目标。

%%% Local Variables:
%%% mode: latex
%%% TeX-master: "../../book"
%%% End: