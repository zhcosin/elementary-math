
\section{题集}
\label{sec:number-sequence-exercise}
\begin{exercise}
  已知各项都是正实数的数列$x_n$对一切正整数$n$都成立$x_n+\frac{1}{x_{n+1}}<2$,求证该数列所有项都满足$x_n<1$.
\end{exercise}
\begin{proof}[解答]
  如果用上极限理论,则可以很容易的得出它单调增加并以1为极限,结论不证自明,所以这里主要讨论的是初等证明。
因为
$$
x_n+\frac{1}{x_{n+1}}<2 \leqslant 
x_{n+1}+\frac{1}{x_{n+1}}
$$
所以$x_n<x_{n+1}$,即该数列单调增加。
又显然$x_n<2$,所以
$$
2>x_n+\frac{1}{x_{n+1}}>x_n+\frac{1}{2}
$$
于是$x_n<2-\frac{1}{2}$,我们得到一个更加好的上限,重复这个过程,我们由$x_n<y_m$就可以得到
$$
x_n<2-\frac{1}{y_m}
$$
所以我们作数列$y_m$,它由$y_0=2$和
$$
y_{m+1}=2-\frac{1}{y_m}
$$
来确定。
数列$y_m$的每一项都大过数列$x_n$的全部项,所以它的下标特意用$m$而不是$n$来表示,以示不相关。
以下我们来证明,数列$y_m$都大于1但是可以任意接近1,只要证明了这一点,我们就有$x_n\leqslant 1$,而$x_n$的单调性保证了等号是不能取的。
$y_m>1$这一点根据数学归纳法是明显成立的。下面证明它可以任意接近1,也就是要证明,对于无论多么小的正实数$\delta$,总存在$y_m$中的某一项$y_M$,使得$y_M<1+\delta$。
直接证明好像还不是太容易,倒是反证法方便些,假定存在某个正实数$\delta$,使得$y_m$中的所有项都满足$y_m\geqslant 1+\delta$,则
$$
y_{m+1}-1=\frac{1}{y_m}(y_m-1)\leqslant 
\frac{1}{1+\delta}(y_m-1)
$$
于是
$$
y_m-1\leqslant \frac{1}{(1+\delta)^m}
$$
显然与假设矛盾,故得证。
\end{proof}


%%% Local Variables:
%%% mode: latex
%%% TeX-master: "../../book"
%%% End:
