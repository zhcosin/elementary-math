
\section{多项式}
\label{sec:polynome}

\subsection{整除的概念与带余除法}
\label{sec:polynome-integer-division-and-devision-with-remainder}

\subsection{最大公因式与辗转相除法}
\label{sec:greatest-common-divisor-and-euclidean-division}

\subsection{因式分解定理}
\label{sec:factoring-theorem}

(包含复系数与实系数多项式的因式分解)

\subsection{重因式}
\label{sec:mulitple-factor}

\subsection{多项式函数}
\label{sec:polynome-function}

在此讨论一下有理系数多项式的有理根的问题,因为有理系数多项式方程总可以化为一个整系数多项式方程,所以只要讨论整系数多项式的根就行了,这时我们有以下定理
\begin{theorem}
  整系数$n$次多项式
  \[ a_nx^n+a_{n-1}x^{n-1}+\cdots+a_1x+a_0 \]
  如果有一个有理根$r/s$($r$、$s$互素),则$r \mid a_0$,$s \mid a_n$,在最高次项系数$a_n=1$的特殊情况下,它的有理数都只能是整数根,而且这些根都是常数项$a_0$的因数。
\end{theorem}

\begin{proof}[证明]
  由条件得
  \[ a_n \left( \frac{r}{s} \right)^{n} + a_{n-1}\left( \frac{r}{s} \right)^{n-1} + \cdots + a_1 \frac{r}{s} + a_0 = 0 \]
  整理得
  \[ a_nr^n + a_{n-1}r^{n-1}s + \cdots + a_1rs^{n-1} + a_0s^n = 0 \]
  左边除最后一项外都能被$r$整除,所以$r \mid a_0s^{n-1}$,而$r$与$s$互素,所以$r \mid a_0$,同样,左边除第一项外都能被$s$整除,所以$s \mid a_nr^n$,从而$s \mid a_n$.
\end{proof}

\subsection{插值多项式}
\label{sec:interpolation-polynome}






%%% Local Variables:
%%% mode: latex
%%% TeX-master: "../../book"
%%% End:
