
\section{线性方程组}
\label{sec:linear-equations-system}

\begin{example}[卫星定位原理]
  先讨论下空间定位问题,假使测点$P(x,y,z)$到一组参考点$A_i(x_i,y_i,z_i)$的距离是$d_i$,$n=1,2,\ldots,n$, 那么测点$P$必同时位于以参考点$P_i$为球心且半径为$d_i$的$n$个球面上。一般而言,至少要有四个参考点才能唯一确定测点的位置,并且这四个参考点还不能共面。

  容易写出测点$P$坐标所满足的四个球面方程:
  \[ E_i: \quad (x-x_i)^2+(y-y_i)^2+(z-z_i)^2=d_i^2, i = 1,2,3,4 \]
  剩下的问题就是如何求解这个方程组了,注意到这四个方程的二次项系数都相同,任何两个方程相减都会得出一个三元一次方程,因而可以转化为线性方程组加以解决. 这四个方程两两组合可以得出六个一次方程,通过消元法可以直接解出这个方程组.
  
  注意到实际上未知数只有三个,所以这六个方程,有三个是多余的,换句话说,该方程组的增广矩阵的秩为3,虽然方程组可以直接求解,但此处对于如何从六个方程选出三个提供另一种视角.

 把这四个参考点看成一个四面体,则方程$E_i(i=1,2,3,4)$分别对应四面体的四个顶点,用 $E_i$和$E_j$相减得出的三元一次方程则相当于棱$E_iE_j$,这样,从六个方程组中挑选三个方程就相当于在四面体的六条棱中选出三条棱。这里先给出结论:任意不共面的三条棱均符合要求。一个基本事实是,不能选择共面的三条棱,因为同一个面上的两条棱实际上是能推出第三条棱的,因为对于某个面上的三个方程$E_iE_j$、$E_jE_k$、$E_kE_i$,显然其中任意两个方程相加减可以作出第三个方程。有了这个基本事实,就很容易证明从任意不共面的三条棱出发,都能推出其余三条棱,这样就证实了我们的结论。

 对于卫星定位而言,通常使用的是经纬坐标系统,经纬坐标只能刻画地球表面的位置,如果再增加一个参数:到地球球心的距离,便可以将经纬系统推广到整个宇宙空间,这时,宇宙空间中任一点的经纬度,就是它与地心连线与地面交点的经纬度。为将经纬坐标转换为空间直角坐标,以地球球心为坐标系原点,零度经线与赤道相交于 $X$,则取 $\vv{OX}$ 为 $x$ 轴,东经90度经线与赤道相交于 $Y$,取 $\vv{OY}$ 为 $y$ 轴,地球北极为 $N$,取 $\vv{ON}$ 为 $z$ 轴,地球半径记为 $R$,到地球球心的距离记为 $r$,那么根据球面参数方程有.
 \[
   \begin{cases}
     x = r \cos{\beta}\cos{\alpha} \\
     y = r \cos{\beta}\sin{\alpha} \\
     z = r \sin{\beta}
   \end{cases}
 \]
 式中,$\alpha$ 与 $\beta$ 分别代表经度和纬度,对于地球表面的点而言,只要将 $r$ 换成地球半径即可. 

 据此,只要知道四个参考点的经纬度坐标,以及测点到各参考点的距离,便能求得测点的空间坐标,并进而将此空间坐标转换为经纬度坐标加上到球心的距离,具体的转换是
 \[ \cos{\beta} = \frac{\sqrt{x^2+y^2}}{\sqrt{x^2+y^2+z^2}}, \sin{\beta} = \frac{z}{\sqrt{x^2+y^2+z^2}} \]
 以及
 \[ \cos{\alpha} = \frac{x}{\sqrt{x^2+y^2}}, \sin{\alpha} = \frac{y}{\sqrt{x^2+y^2}} \]
 所以完整的定位流程是:
 \begin{enumerate}
 \item 预先得到各参考点(至少四个)的经纬度坐标
 \item 测量测点到各参考点的距离
 \item 求得测点的空间坐标,再转换为经纬度坐标
 \end{enumerate}
 这只是最基础的原理,有大量的实际问题并未考虑在内,首先就是如何确定各参考点的经纬坐标,其次如何得到测点到各参考点的距离,通过发送电磁信号并测量传播用时的方法,又涉及到一个高科技问题,因为电磁信号传播速度为光速,那么时间精度便是一个难题。此外,限于国土面积限制,如何把参考点都设在国内,那么定位服务将仅限于自己国家范围,如果在海外基地或者测量船上搭建移动基站,那么还涉及到移动参考点问题,诸如此类,这是一个非常复杂的系统工程。
 \end{example}
%%% Local Variables:
%%% mode: latex
%%% TeX-master: "../../elementary-math-note"
%%% End:
