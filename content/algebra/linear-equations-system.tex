
\section{线性方程组}
\label{sec:linear-equations-system}

\begin{example}[简单的三维定位]
  简单讨论下空间定位问题,假使测点$P(x,y,z)$到一组参考点$A_i(x_i,y_i,z_i)$的距离是$d_i$,$n=1,2,\ldots,n$, 那么测点$P$必同时位于以参考点$P_i$为球心且半径为$d_i$的球面上。容易知道,至少要有四个参考点才能唯一确定测点的位置,并且这四个参考点还不能共面。

  容易写出测点$P$坐标所满足的四个球面方程:
  \[ E_i: \quad (x-x_i)^2+(y-y_i)^2+(z-z_i)^2=d_i^2, i = 1,2,3,4 \]
  剩下的问题就是如何求解这个方程组了,注意到这四个方程的二次项系数都相同,任何两个方程相减都会得出一个三元一次方程,因而可以转化为线性方程组加以解决. 这四个方程两两组合可以得出六个一次方程,但实际上未知数只有三个,所以这六个方程,有三个是多余的,换句话说,该方程组的增广矩阵的秩为3.

  用 $E_i$ 和 $E_j$两个方程相减得到的三元一次方程代表着一张平面,它就是以$P_i$为球心半径为$d_i$,与以$P_j$为球心半径为$d_j$的两个球面的交线圆所在平面(垂直于两个球心连线). 从这个意义上说,如果我们把这四个参考点看成一个四面体,则方程$E_i(i=1,2,3,4)$分别对应四面体的四个顶点,用 $E_i$和$E_j$相减得出的三元一次方程则相当于棱$E_iE_j$,从这个意义上说,对于同一个面上的三条棱,任意两个都能得出第三个,所以为了从六个方程中选出三个进行求解,这三个方程对应的棱不能在同一个面上。
\end{example}
%%% Local Variables:
%%% mode: latex
%%% TeX-master: "../../elementary-math-note"
%%% End:
