
\section{线性方程组}
\label{sec:linear-equations-system}

\begin{example}[简单的三维定位]
  简单讨论下空间定位问题,假使测点$P(x,y,z)$到一组参考点$A_i(x_i,y_i,z_i)$的距离是$d_i$,$n=1,2,\ldots,n$, 那么测点$P$必同时位于以参考点$P_i$为球心且半径为$d_i$的$n$个球面上。容易知道,至少要有四个参考点才能唯一确定测点的位置,并且这四个参考点还不能共面。

  容易写出测点$P$坐标所满足的四个球面方程:
  \[ E_i: \quad (x-x_i)^2+(y-y_i)^2+(z-z_i)^2=d_i^2, i = 1,2,3,4 \]
  剩下的问题就是如何求解这个方程组了,注意到这四个方程的二次项系数都相同,任何两个方程相减都会得出一个三元一次方程,因而可以转化为线性方程组加以解决. 这四个方程两两组合可以得出六个一次方程,通过消元法可以直接解出这个方程组.
  
  注意到实际上未知数只有三个,所以这六个方程,有三个是多余的,换句话说,该方程组的增广矩阵的秩为3,虽然方程组可以直接求解,但此处对于如何从六个方程选出三个提供另一种视角.

 把这四个参考点看成一个四面体,则方程$E_i(i=1,2,3,4)$分别对应四面体的四个顶点,用 $E_i$和$E_j$相减得出的三元一次方程则相当于棱$E_iE_j$,这样,从六个方程组中挑选三个方程就相当于在四面体的六条棱中选出三条棱。这里先给出结论:任意不共面的三条棱均符合要求。一个基本事实是,不能选择共面的三条棱,因为同一个面上的两条棱实际上是能推出第三条棱的,因为对于某个面上的三个方程$E_iE_j$、$E_jE_k$、$E_kE_i$,显然其中任意两个方程相加减可以作出第三个方程。有了这个基本事实,就很容易证明从任意不共面的三条棱出发,都能推出其余三条棱,这样就证实了我们的结论。
\end{example}
%%% Local Variables:
%%% mode: latex
%%% TeX-master: "../../elementary-math-note"
%%% End:
