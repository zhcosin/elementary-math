
\section{用数列方法解决错位排列问题}
\label{sec:derangement-problem}

此文要解决的问题是:有编号为$1,2,\ldots,n$的$n$个人和同样编号的$n$个座位,如果每个人都不允许坐与自身编号相同的座位,有多少种坐法?此即所谓的错位排列问题。

这个用容斥原理是可以瞬间解决的,高中时期年少轻狂,愣是用定义数列求通项的方法给求出来了,此文即是重新写的,原稿已经遗失。

考虑编号为1的人,假如他坐在编号为$r_1$的座位上,又看编号为$r_1$的人,他可以坐在1号座位上,也可以坐在$r_2$号座位上,如果是坐在1号座位上,那么1号和$r_1$号这两个人由于交叉坐,与其他人无涉,如果是坐在$r_2$号座位上,则再继续看$r_2$号人,依此顺着链条$1\rightarrow r_1 \rightarrow r_2 \rightarrow \cdots$,这样下去的结果是,从1号人开始,能够找到一个环,使得该环内的人和座位与环外的人和座位无涉,可以独立开来,如果找不到这样的环,那无非是所有人一起构成了一个整体环,即是无法分割。这样的环可能不止一个,但我们仅考虑包含1号人的那个环。

假定$n$个人的错位排列数为$a_n$,设包含1号人的环共含有$m(2\leqslant m \leqslant n-2)$个人,那么只要从除1号之外的人中选出$m-1$个人排在这个座位链上,因此座法是$(m-1)!C_{n-1}^{m-1}=\frac{(n-1)!}{(n-m)!}$,而剩下的$n-m$个人又组成了一个较小的错位排列,因此其坐法数是$a_{n-m}$,此外,当$m=n$时,所有人构成一个环,只要把他们按照座位链排成一排即可,所以此时的坐法是$(n-1)!$,于是总共的坐法是:
\begin{equation*}
  a_n=\sum_{m=2}^{n-2}\frac{(n-1)!}{(n-m)!}a_{n-m}+(n-1)!=(n-1)!(\sum_{m=2}^{n-2}\frac{a_{n-m}}{(n-m)!}+1)
\end{equation*}
将式中的求和顺序倒过来,就有
\begin{equation}
  \label{eq:requsive-equation-derangement}
  a_n=(n-1)!(\sum_{m=2}^{n-2}\frac{a_{m}}{m!}+1)
\end{equation}
此即其作为数列的递推公式,自然的,$a_1=0,a_2=1$。

下面来求它的通项公式,上式可以变形为:
\begin{equation}
  \label{recursive-equation-derangement-2}
  \frac{a_n}{n!}=\frac{1}{n}(1+\sum_{m=2}^{n-2}\frac{a_m}{m!})
\end{equation}
只要记$b_n=\frac{a_n}{n!}$,则有$b_1=0$和
\begin{equation}
  \label{eq:recursive-equation-derangement-simple}
  b_n=\frac{1}{n}(1+\sum_{m=2}^{n-2}b_m)
\end{equation}
于是有
\begin{equation}
  \label{eq:recursive-equation-derangement-simple-diff}
  nb_n-(n-1)b_{n-1}=b_{n-2}
\end{equation}
也就是
\begin{equation}
  \label{eq:recursive-equation-derangement-simple-diff2}
  b_n-b_{n-1}=-\frac{1}{n}(b_{n-1}-b_{n-2})
\end{equation}
而$b_2-b_1=\frac{1}{2}$,所以
\begin{equation}
  \label{eq:recursive-equation-derangement-bn}
  b_n-b_{n-1}=(-1)^{n-2}\frac{1}{n!}=(-1)^n\frac{1}{n!}
\end{equation}
于是
\begin{equation}
  \label{eq:derangement-bn}
  b_n=b_1+\sum_{m=2}^{n}(b_m-b_{m-1})=\sum_{m=2}^n(-1)^m\frac{1}{m!}
\end{equation}
所以最终
\begin{equation}
  \label{eq:derangement-an}
  a_n=n!b_n=n!\sum_{m=2}^n(-1)^m\frac{1}{m!}
\end{equation}
在$0!=1$的规定下,也可以把求和指标从零开始
\begin{equation}
  \label{eq:derangement-an2}
  a_n=n!\sum_{m=0}^n(-1)^m\frac{1}{m!}
\end{equation}
这就是最终的结果。

%%% Local Variables:
%%% mode: latex
%%% TeX-master: "../../elementary-math-note"
%%% End:
