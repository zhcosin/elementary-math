
\section{曲线、曲面与方程}
\label{sec:curve-and-equation}

在笛卡尔坐标几何创立之前,几何学与代数学是两个研究对象和研究方法都各自独立的学科,彼此之间并无多少实质性的联系,几何学历来以难著称,层出不穷的几何证明技巧,惊为神来之笔的各种几何辅助线给人以深刻的印象,但是一直以来,缺乏一种通用的或者说万能的方法来处理几何问题。

古希腊数学家阿波罗尼奥斯的名著《圆锥曲线论》将古典几何推向了一个巅峰,该书利用几何方法将圆锥曲线的性质几乎一网打尽,以致于后人在长达两千余年间没能在这个领域有多少重大发现。但笛卡尔的坐标几何改变了这一点。在笛卡尔的坐标思想中,利用坐标来刻画点的位置,于是位置、距离、角度等几何量统统被坐标量化,于是产生了一种新的研究几何学的方法,就是利用纯粹的代数运算来证明几何性质,这就是 \emph{解析几何},利用这种新的方法,人们又发现了圆锥曲线的一些新的性质,对圆锥曲线的研究才又有了新的突破性发展。

\subsection{曲线与曲面的方程、方程的曲线或曲面}
\label{sec:equation-of-curve-and-curve-of-equation}

今后所称曲线,是通指线型的几何图形,包括直线,只是按一般情形进行通称,并不意味曲线是弯曲的。

\begin{definition}
  在建立了直角坐标系的平面中,对于曲线$C$和方程$f(x,y)=0$,如果
  \begin{enumerate}
  \item 曲线上任一点$P(x_P,y_P)$,其坐标都满足方程$f(x,y)=0$,即$f(x_P,y_P)=0$.
  \item 任一个坐标满足方程$f(x,y)=0$的点$P(x_p,y_p)$都在曲线$C$上.
  \end{enumerate}
  则称方程$f(x,y)=0$是曲线$C$的方程,而曲线$C$是方程$f(x,y)=0$的曲线。
\end{definition}

类似的有空间曲面的方程,与方程的曲面的概念,这里的曲面同样也包括平面.

\begin{definition}
  在空间直角坐标系中,对于曲面$C$和方程$f(x,y,z)=0$,如果
  \begin{enumerate}
  \item 曲面上任一点$P(x_P,y_P,z_P)$,其坐标都满足方程$f(x,y,z)=0$,即$f(x_P,y_P,z_p)=0$.
  \item 任一个坐标满足方程$f(x,y,z)=0$的点$P(x_p,y_p,z_p)$都在曲面$C$上.
  \end{enumerate}
  则称方程$f(x,y,z)=0$是曲面$C$的方程,而曲面$C$是方程$f(x,y,z)=0$的曲面。
\end{definition}

方程$f(x,y)=0$或者$f(x,y,z)=0$称为曲线或曲面的 \emph{普通方程},实际上很多曲线难以用普通方程表达出来,而更常用的是 \emph{参数方程}.

\begin{definition}
  如果曲线$C$上任一点的坐标,都是某个数$t$的函数,即
  \[ \left\{
      \begin{array}{lll}
        x & = & x(t) \\
        y & = & y(t) \\
        z & = & z(t)
      \end{array}
    \right. \]
  则称该方程(组)是曲线$C$的\emph{参数方程},类似的,如果曲面上任一点的坐标,都是两个参数$t$和$s$的二元函数,即
  \[ \left\{
      \begin{array}{lll}
        x & = & x(t, s) \\
        y & = & y(t, s) \\
        z & = & z(t, s)
      \end{array}
    \right. \]
  则称该方程(组)是曲面的 \emph{参数方程}.
\end{definition}

\subsection{直线和平面的方程}
\label{sec:equation-of-line-and-plane}

先来考虑平面上直线的方程,为此先给出直线的方向向量和法向量的定义.

\begin{definition}
  能够与直线上两点所确定的向量共线的非零向量,称为该直线的 \emph{方向向量},而能与直线的方向向量垂直的向量,称为该直线的 \emph{法向量}。
\end{definition}

\begin{definition}
  对于空间中的一张平面,凡能与平面内两点所确定的向量共线的非零向量称为该平面的\emph{方向向量},即由方向向量决定的直线族与平面只能是平行或包含的关系。
\end{definition}

\begin{definition}
  空间中,如果直线与平面垂直,则直线称为平面的\emph{法线},而平面称为直线的\emph{法平面},与平面的法线共线的向量称为平面的\emph{法向量}。
\end{definition}

方向向量和法向量均唯一的确定了直线的走向。

为了确定直线,除了直线的走向,还需要确定直线的位置,于是再给出直线上的一个点就行了,所以先来考虑,给定直线$l$的方向向量$\bm{v}=(a,b)$及直线上一点$P(x_0,y_0)$的情形下,直线的方程。

设直线上任一点的坐标是$Q(x,y)$,于是有$\vv{PQ}$与方向向量$\bm{v}$共线,于是按照\autoref{theorem:collinear-vector-codrnation}在平面上的结论,就有
\[ \frac{x-x_0}{a} = \frac{y-y_0}{b} \]
显然,凡满足此方程的任一对坐标所对应的点,其与点$P$所构成的向量也与方向向量$\bm{v}$共线,从而也必然在直线$l$上,因此这个方程就是所要求的,它称为直线的 \emph{点向式方程},在分母为零时约定分子也为零。

同理,对于空间中由方向向量$\bm{v}=(a,b,c)$及直线上一点$P(x_0,y_0,z_0)$所确定的直线的方程是
\[ \frac{x-x_0}{a} = \frac{y-y_0}{b} = \frac{z-z_0}{c} \]
显然这并不是一个方程,而是一个方程组,这就是说,空间中的直线,需要两个三元一次方程来确定。

因为直线只需要两个点就可以唯一确定,所以考虑在平面上由两个点$A(x_1,y_1)$和$B(x_2,y_2)$所确定的直线的方程,因为这时方向向量是$\vv{AB}=(x_1-x_2,y_1-y_2)$,所以应用刚才得出的点向式方程(同时还过点$A$),得出直线方程是
\[ \frac{x-x_1}{x_1-x_2} = \frac{y-y_1}{y_1-y_2} \]
这便是直线的\emph{两点式方程},一个特殊情况是,经过$A(a,0)$和$B(0,b)$两点的直线方程是
\[ \frac{x}{a}+\frac{y}{b}=1 \]
这里的$A$和$B$两点分别是直线与两个坐标轴的交点,而数$a$和$b$分别称为直线在两个坐标轴上的\emph{截距}(即截点到原点的代数距离),所以这方程称为直线的\emph{截距式方程}.

同理可得空间中,经过两点$A(x_1,y_1,z_1)$和$B(x_2,y_2,z_2)$的直线方程是
\[ \frac{x-x_1}{x_1-x_2} = \frac{y-y_1}{y_1-y_2} = \frac{z-z_1}{z_1-z_2} \]

接着再来考虑由直线上一点$P(x_0,y_0)$及它的法向量$\bm{n}=(a,b)$所确定的直线的方程,设直线上任一点$Q(x,y)$,则有$\bm{n} \perp \vv{PQ}$,所以$\bm{n} \cdot \vv{PQ} = 0$,即
\[ a(x-x_0)+b(y-y_0)=0 \]
这就是所要求的,它称为直线的 \emph{点法式方程}.

对于空间中的直线,显然一条法向量并不足以确定它的走向,需要两个不共线的法向量(均与法平面垂直,法平面是指与直线垂直的平面),设两个法向量分别是$\bm{n_1}=(a_1,b_1,c_1)$和$\bm{n_2}=(a_2,b_2,c_2)$,同样求内积为零可得出
\[ \left\{
    \begin{array}{lll}
      a_1(x-x_0) + b_1(y-y_0) + c_1(z-z_0) & = & 0 \\
      a_2(x-x_0) + b_2(y-y_0) + c_2(z-z_0) & = & 0
    \end{array}
  \right. \]
这再一次验证了空间直线需要两个三元一次方程才能确定,实际上接下来会看到,一个三元一次方程代表空间中的一张平面,而直线则是两个平面的交线。

还是回头考虑由直线上一点$P(x_0,y_0)$和直线的方向向量$\bm{v}=(a,b)$所确定的直线,对于直线上任一点$Q(x,y)$,由$\bm{v}$与$\vv{PQ}$共线,存在唯一实数$t$,使得$\vv{PQ}=t\bm{v}$,展开坐标就是
\[ \left\{
    \begin{array}{lll}
      x & = & x_0 + at \\
      y & = & y_0 + bt
    \end{array}
  \right. \]
可见,直线上任一点均由参量$t$的不同取值决定,$t$的取值与直线上的点构成一一对应的关系,所以这个方程组就是直线的 \emph{参数方程}.

同样,对于空间直线上任一点$P(x_0,y_0,z_0)$及其方向向量$\bm{v}=(a,b,c)$所确定的直线的参数方程是
\[ \left\{
    \begin{array}{lll}
      x & = & x_0 + at \\
      y & = & y_0 + bt \\
      z & = & z_0 + ct
    \end{array}
  \right. \]

接着讨论空间平面的方程,类似于平面上的直线,平面由其法向量$\bm{n}=(a,b,c)$及平面上一点$P(x_0,y_0,z_0)$所唯一确定,设$Q(x,y)$是该平面上任一点,则$\bm{n} \perp \vv{PQ}$,于是$\bm{n} \cdot \vv{PQ}=0$,所以
\[ a(x-x_0)+b(y-y_0)+c(z-z_0)=0 \]
这称为平面的 \emph{点法式方程}。

如果要用方向向量而不是法向量,则平面需要两个不共线的方向向量才能唯一确定平面的倾斜方向,设两个方向向量是$\bm{v_1}=(a_1,b_1,c_1)$和$\bm{v_2}=(a_2,b_2,c_2)$,则对于平面上任一点$Q(x,y,z)$,有$\vv{PQ}$与$\bm{v_1}$及$\bm{v_2}$共面,由\autoref{theorem:coplanear-vector-cordination}可得
\[
  \begin{vmatrix}
    x-x_0 & y-y_0 & z-z_0 \\
    a_1 & b_1 & c_1 \\
    a_2 & b_2 & c_2
  \end{vmatrix}
  =0
\]
或者展开成为
\[
  \begin{vmatrix}
    b_1 & c_1 \\
    b_2 & c_2
  \end{vmatrix}
  (x-x_0) -
  \begin{vmatrix}
    a_1 & c_1 \\
    a_2 & c_2
  \end{vmatrix}
  (y-y_0) +
  \begin{vmatrix}
    a_1 & b_1 \\
    a_2 & b_2
  \end{vmatrix}
  (z-z_0) = 0
\]
这两个方程就是平面的 \emph{点向式方程}.

而由$\vv{PQ}$与$\bm{v_1}$及$\bm{v_2}$共面,存在唯一一对实数$u$和$v$,使得$\vv{PQ}=u \bm{v_1} + v \bm{v_2}$,写成坐标形式就是
\[
  \left\{
    \begin{array}{lll}
      x & = & x_0 + u a_1 + v a_2 \\
      y & = & y_0 + u b_1 + v b_2 \\
      z & = & z_0 + u c_1 + v c_2
    \end{array}
    \right.
\]
这就是平面的 \emph{参数方程}.

在上面看到,平面上的直线方程都是二元一次方程,空间中的平面方程也都是三元一次方程,那反过来,是否每一个二元一次方程和每一个三元一次方程都分别代表二维平面上的一条直线和空间中的一张平面呢?

二元一次方程的一般形式是$Ax+By+C=0$,其中$A$、$B$不同时为零,假定$A \neq 0$,则点$P(-\dfrac{C}{A},0)$在这方程所表示的图形上,于是方程可改写为
\[ A(x+\frac{C}{A})+B(y-0)=0 \]
显然这表示通过点$P$并以$\bm{n}=(A,B)$为法向量的直线,所以它必然表示一条直线,方程$Ax+By+C=0$便称为平面上直线的 \emph{一般方程}。

同样,在空间中,三元一次方程$Ax+By+Cz+D=0$中,$A$、$B$、$C$不同时为零,假定$A \neq 0$,于是它便是通过点$P(-\dfrac{D}{A},0,0)$并以$\bm{n}=(A,B,C)$为法向量的平面,它便称为空间平面的 \emph{一般方程}.

\subsection{圆和球面的方程}
\label{sec:equation-of-circle-and-ball}

在平面直角坐标系中,设某圆的圆心坐标坐标是$C(a,b)$,半径为$r$,则对于圆上任一点$P(x,y)$,有$|PC|=r$,即$\sqrt{(x-a)^2+(y-b)^2}=r$,这等价于
\[ (x-a)^2+(y-b)^2=r^2 \]
反过来,如果某个点$Q$的坐标满足上述方程,则两边开方就得出$|QC|=r$,即点$Q$必在此圆上,所以这个方程就是以点$(a,b)$为圆心,$r$为半径的圆的方程,它称为圆的 \emph{标准方程}.

将圆的标准方程展开,得到一个关于$x$和$y$的二元二次方程
\[ x^2+y^2-2ax-2ay+a^2+b^2-r^2=0 \]
这方程具有几个特点:(I)$x^2$与$y^2$项系数相同,(II)不含$xy$项. 对于满足这两个条件的一般的二元二次方程(将$x^2$和$y^2$系数化为1)
\[ x^2+y^2+Dx+Ey+F=0 \]
它可以配方成为
\[ \left( x+\frac{D}{2} \right)^2 + \left( y+\frac{E}{2} \right)^2 = \frac{1}{4}(D^2+E^2-4F) \]
显然在$D^2+E^2-4F>0$时它表示以$\left( -\dfrac{D}{2}, -\dfrac{E}{2} \right)$为圆心,以$\sqrt{D^2+E^2-4F}$为半径的圆,于是一般的二元二次方程需要满足三个条件才能成为圆的方程,上面特殊形式的二元二次方程称为圆的 \emph{一般方程}。

类似的可以得到,空间中以点$C(a,b,c)$为球心,$r$为半径的球面方程是
\[ (x-a)^2+(y-b)^2+(z-c)^2=r^2 \]

对于以$C(a,b)$为圆心,$r$为半径的圆,由圆心沿$x$轴正方向所确定的点是$A(a+r,b)$,对圆心上其它任意一点$Q(x,y)$,可以视为由点$A$绕圆心旋转一定的角度$\theta$而得到,于是按三角函数的定义有$\cos{\theta}=\dfrac{x-a}{r},\sin{\theta}=\dfrac{y-b}{r}$,于是得到
\[ \left\{
    \begin{array}{lll}
      x & = & a + r \cos{\theta} \\
      y & = & b + r \sin{\theta} 
    \end{array}
  \right. \]
这就是圆的 \emph{参数方程}.

再来考虑球面的参数方程,设球心是$C(a,b,c)$,半径为$r$,过球心作$xOy$平面的平行平面$\alpha$,点$A(a+r,b,c)$是该平面与球面交线上的一个点,设$Q(x,y,z)$是球面上任一点,设射线$OQ$与平面$\alpha$的夹角是$\theta$,过点$Q$再作平面$xOy$的平行平面$\beta$,与球面交成一个纬线圈,则点$Q$可由点$A$经过两次旋转而得到,第一次旋转由点$A$出发沿着经线圈旋转一个角度到达点$Q$所在的纬线圈上的点$M$,显然旋转角度就是$\theta$,再由$M$出发沿着该纬线圈旋转一个角度$\varphi$到达点$Q$,于是点$Q$所决定的纬线圈的圆心坐标是$K(a,b,c+r\sin{\theta})$,半径是$r\cos{\theta}$,于是由圆的参数方程有
\[ \left\{
    \begin{array}{lll}
      x_Q & = & x_k + r \cos{\theta}\cos{\varphi}  \\
      y_Q & = & y_k + r \cos{\theta}\sin{\varphi} \\
      z_Q & = & z_k
    \end{array}
  \right. \]
于是得出球面的参数方程是
\[ \left\{
    \begin{array}{lll}
      x_Q & = & a + r \cos{\theta}\cos{\varphi}  \\
      y_Q & = & b + r \cos{\theta}\sin{\varphi} \\
      z_Q & = & c+r\sin{\theta}
    \end{array}
  \right. \]



%%% Local Variables:
%%% mode: latex
%%% TeX-master: "../../elementary-math-note"
%%% End:
