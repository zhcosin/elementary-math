
\section{三面角与四面体}
\label{sec:trihedral-angle}

在空间中,从一点引出三条射线组成了一个有趣的模型,它包含了三个角,以及三个扇形平面区域,还有三个二面角,如果在三条射线上各取一个点并互相连接起来,就得到一个四面体,三面角与四面体在空间几何中,与三角形在平面几何中类似,简单的图形却具有丰富的内容.

\subsection{三面角的余弦公式}
\label{sec:cosin-of-three-side-angle}


现在我们感兴趣的是,这三条射线的两两夹角,与三个面形成的三个二面角之间的关系。

为研究这个关系,我们提炼出下面这样的一个简化模型:


\begin{figure}[htbp]
\centering
\includegraphics{content/plane-geometry/pic/cosin-of-three-side-angle.pdf}
\caption{三面角余弦定理}
\label{fig:cosin-of-three-side-angle}
\end{figure}

如图 \autoref{fig:cosin-of-three-side-angle} 所示, 从一个二面角的棱上某点 $O$ 分别向两个半平面内各引一条射线,如果我们固定住这两条射线与棱之间的夹角,而让二面角的大小可以自由变化,那么显然这两条射线的夹角就只跟这个二面角的大小有关了。

在棱上取异于 $O$ 的一点 $H$,过点$H$分别在两个半平面内作棱的垂线,分别与两条射线相交于 $A$ 和 $B$,记 $\angle AOH = \alpha$, $\angle BOH=\beta$,二面角 $AHB=\theta$,我们来求 两条射线的夹角 $\gamma = \angle AOB$.

约定 $OH = 1$,有
\[ AO = \frac{1}{\cos{\alpha}}, BO = \frac{1}{\cos{\beta}}, AH = \tan{\alpha}, BH = \tan{\beta} \]
由余弦定理可得 $AB$长度为
\[ AB^{2} = AH^2+BH^2-2 \cdot AH \cdot BH \cdot \cos{\theta} \]
代入各式可得
\[ AB^2 = \tan^2{\alpha} + \tan^2{\beta} - 2 \tan{\alpha} \tan{\beta} \cos{\theta} \]
再由余弦定理可得
\[ \cos{\gamma} = \frac{AO^2+BO^2-AB^2}{2 \cdot AO \cdot BO} \]
代入各式最终得
\[ \cos{\gamma} = \cos{\alpha}\cos{\beta}+\sin{\alpha}\sin{\beta}\cos{\theta} \]
这便是最终结果,或者写成由三条射线夹角表达二面角的形式:
\[ \cos{\theta} = \frac{\cos{\gamma}-\cos{\alpha}\cos{\beta}}{\sin{\alpha}\sin{\beta}} \]

在这结果中分别取 $\theta = \pi$ 和 $\theta = 0$,我们就分别得到余弦的和角公式和差公式:
\begin{eqnarray*}
  \cos{(\alpha+\beta)} & = & \cos{\alpha}\cos{\beta} - \sin{\alpha}\sin{\beta} \\
  \cos{(\alpha-\beta)} & = & \cos{\alpha}\cos{\beta} + \sin{\alpha}\sin{\beta} 
\end{eqnarray*}
另外一个特殊的情况是,当二面角为直二面角时,有
\[ \cos{\gamma} = \cos{\alpha}\cos{\beta} \]

有了这个结果,对于前述空间中一点引出三条射线的模型,三条射线的夹角,与三个二面角之间的关系就清楚了。

\begin{example}[根据经纬度计算地球上两点间的(球面)距离]
  在这个例子中,我们要研究的是,已知地球上两点的经纬度坐标,如果计算出这两点间的球面距离。
\end{example}


%%% Local Variables:
%%% mode: latex
%%% TeX-master: "../../elementary-math-note"
%%% End:
