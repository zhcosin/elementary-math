
\section{三面角与四面体}
\label{sec:trihedral-angle}

在空间中,从一点引出三条射线组成了一个有趣的模型,它包含了三个角,以及三个扇形平面区域,还有三个二面角,这个图形我们称为 \emph{三面角}. 三面角的三条射线两两所成角称为它的 \emph{面角}. 容易知道,它的三个面角之间存在着类似于三角形三边的和差关系:任意两个面角之和不小于第三角,任意两个面角之差不大于第三个面角.


\subsection{三面角的余弦公式}
\label{sec:cosin-of-three-side-angle}


现在我们感兴趣的是,这三条射线的两两夹角,与三个面形成的三个二面角之间的关系。

为研究这个关系,我们提炼出下面这样的一个简化模型:


\begin{figure}[htbp]
\centering
\includegraphics{content/plane-geometry/pic/cosin-of-three-side-angle.pdf}
\caption{三面角余弦定理}
\label{fig:cosin-of-three-side-angle}
\end{figure}

如图 \autoref{fig:cosin-of-three-side-angle} 所示, 从一个二面角的棱上某点 $O$ 分别向两个半平面内各引一条射线,如果我们固定住这两条射线与棱之间的夹角,而让二面角的大小可以自由变化,那么显然这两条射线的夹角就只跟这个二面角的大小有关了。

在棱上取异于 $O$ 的一点 $H$,过点$H$分别在两个半平面内作棱的垂线,分别与两条射线相交于 $A$ 和 $B$,记 $\angle AOH = \alpha$, $\angle BOH=\beta$,二面角 $AHB=\theta$,我们来求 两条射线的夹角 $\gamma = \angle AOB$.

约定 $OH = 1$,有
\[ AO = \frac{1}{\cos{\alpha}}, BO = \frac{1}{\cos{\beta}}, AH = \tan{\alpha}, BH = \tan{\beta} \]
由余弦定理可得 $AB$长度为
\[ AB^{2} = AH^2+BH^2-2 \cdot AH \cdot BH \cdot \cos{\theta} \]
代入各式可得
\[ AB^2 = \tan^2{\alpha} + \tan^2{\beta} - 2 \tan{\alpha} \tan{\beta} \cos{\theta} \]
再由余弦定理可得
\[ \cos{\gamma} = \frac{AO^2+BO^2-AB^2}{2 \cdot AO \cdot BO} \]
代入各式最终得
\[ \cos{\gamma} = \cos{\alpha}\cos{\beta}+\sin{\alpha}\sin{\beta}\cos{\theta} \]
这便是最终结果,或者写成由三条射线夹角表达二面角的形式:
\[ \cos{\theta} = \frac{\cos{\gamma}-\cos{\alpha}\cos{\beta}}{\sin{\alpha}\sin{\beta}} \]

在这结果中分别取 $\theta = \pi$ 和 $\theta = 0$,我们就分别得到余弦的和角公式和差公式:
\begin{eqnarray*}
  \cos{(\alpha+\beta)} & = & \cos{\alpha}\cos{\beta} - \sin{\alpha}\sin{\beta} \\
  \cos{(\alpha-\beta)} & = & \cos{\alpha}\cos{\beta} + \sin{\alpha}\sin{\beta} 
\end{eqnarray*}
另外一个特殊的情况是,当二面角为直二面角时,有
\[ \cos{\gamma} = \cos{\alpha}\cos{\beta} \]

有了这个结果,对于前述空间中一点引出三条射线的模型,三条射线的夹角,与三个二面角之间的关系就清楚了。

这里再给出一个这公式的一个向量证明:
\begin{proof}[证明]
  取射线$OA$、$OB$、$OC$三个方向上的单位向量$\bm a$、$\bm b$、$\bm c$,则有
  \[ \bm a\cdot \bm b=\cos{\gamma}, \bm b \cdot \bm c = \cos{\alpha}, \bm c \cdot \bm a = \cos{\beta} \]
  以及
  \[ (\bm b\times \bm c)(\bm c \times \bm a) = \sin{\alpha}\sin{\beta}\cos{\theta} \]
  而根据向量混合积与外积的运算性质,有
  \begin{eqnarray*}
    && (\bm b\times \bm c)(\bm c \times \bm a) \\
    & = & (\bm c \times (\bm c \times \bm a)) \cdot \bm b \\
    & = & ((\bm c \cdot \bm a) \bm c - ( \bm c \cdot \bm c) \bm a) \cdot \bm b \\
    & = & (\cos{\beta} \bm c - \bm a) \cdot \bm b \\
    & = & \cos{\alpha}\cos{\beta}-\cos{\gamma}
  \end{eqnarray*}
  于是便得
  \[ \cos{\alpha}\cos{\beta}-\cos{\gamma} = \sin{\alpha}\sin{\beta}\cos{\theta} \]
\end{proof}

\begin{example}[根据经纬度计算地球上两点间的(球面)距离]
  在这个例子中,我们要研究的是,已知地球上两点的经纬度坐标,如果计算出这两点间的球面距离。
\end{example}

\subsection{三面角的正弦定理}
\label{sec:sine-thoream-of-trihedral-angle}

三面角的三个面角之间有类似于三角形的三边之间的大小关系,那咱们类比一下,三面角的三个面角相当于三角形的三条边,而它的三个二面角相当于三角形的三个内角,那么,在这种类比下,三角形的正弦定理是否可以照搬到三面角中呢,也就是下面的式子是否成立:
\[ \frac{\sin{\alpha}}{\sin{\angle OA}} = \frac{\sin{\beta}}{\sin{\angle OB}} = \frac{\sin{\gamma}}{\sin{\angle OC}} \]
这里的 $\angle OA$、$\angle OB$、$\angle OC$分别指代三条射线处的二面角。

在上一小节我们已经用余弦公式刻画了三面角的二面角是如何由它的三个面角决定的,利用它可以直接写出(注意为了方便,颠倒了一下分子分母)

\begin{eqnarray*}
  \frac{\sin^{2}{\theta}}{\sin^2{\gamma}} & = & \frac{1}{\sin^2{\gamma}} \cdot \left( 1-\frac{\cos{\gamma}-\cos{\alpha}\cos{\beta}}{\sin{\alpha}\sin{\beta}} \right) \\
  & = & \frac{\sin^2{\alpha}\sin^2{\beta}-\cos^2{\gamma}-\cos^2{\alpha}\cos^2{\beta}+2\cos{\alpha}\cos{\beta}\cos{\gamma}}{\sin^2{\alpha}\sin^2{\beta}\sin^2{\gamma}} 
\end{eqnarray*}

到了这里,可以想见,分母以及分子的后半部分已经成为轮换对称式,如果分子的前半部分也能写成关于 $\alpha$、$\beta$、$\gamma$ 的轮换对称式,那么大功就告成了,事实也确实如此:
\begin{eqnarray*}
  && \sin^2{\alpha}\sin^2{\beta}-\cos^2{\gamma}-\cos^2{\alpha}\cos^2{\beta} \\
  & = & (1-\cos^2{\alpha})(1-\cos^2{\beta})-\cos^2{\gamma}-\cos^2{\alpha}\cos^2{\beta} \\
   & = & 1-\cos^2{\alpha}-\cos^2{\beta}-\cos^2{\gamma}
\end{eqnarray*}

这就证实了前面的正弦连等式是成立的,仿照三角形,那式子我们就称为三面角的 \emph{正弦定理},而且还知道了连等式的值:

\begin{eqnarray*}
  && \frac{\sin{\alpha}}{\sin{\angle OA}} = \frac{\sin{\beta}}{\sin{\angle OB}} = \frac{\sin{\gamma}}{\sin{\angle OC}} \\
  & = & \frac{\sin{\alpha}\sin{\beta}\sin{\gamma}}{\sqrt{1-\cos^2{\alpha}-\cos^2{\beta}-\cos^2{\gamma}+2\cos{\alpha}\cos{\beta}\cos{\gamma}}}
\end{eqnarray*}

%%% Local Variables:
%%% mode: latex
%%% TeX-master: "../../elementary-math-note"
%%% End:
