
\section{几个重要的定理}
\label{sec:some-import-plane-geometry-theorem}

本节讲述几个在平面几何中占有重要地位的定理。

\subsection{张角定理}
\label{sec:spread-angle-theorem}

\begin{theorem}[张角定理]
  在三角形$ABC$中,$D$是$BC$边上任意一点,有下式成立:
  \begin{equation}
    \label{eq:spread-angle-theorem}
    \frac{\sin{\angle{BAC}}}{AD} = \frac{\sin{\angle{CAD}}}{AB} + \frac{\sin{\angle{BAD}}}{AC}
  \end{equation}
\end{theorem}

  \begin{figure}[htbp]
  \centering
\includegraphics{content/plane-geometry/pic/spread-angle-theorem.pdf}
\caption{张角定理}
\label{fig:spread-angle-theorem}
\end{figure}

\begin{proof}[证明]
  因为$S_{\triangle ABC} = S_{\triangle ABD} + S_{\triangle ACD}$,所以
  \begin{equation*}
    \frac{1}{2}AB \cdot AC \cdot \sin{\angle BAC} =
    \frac{1}{2}AB \cdot AD \cdot \sin{\angle BAD} +
    \frac{1}{2}AC \cdot AD \cdot \sin{\angle CAD} 
  \end{equation*}
  两边同除以$\frac{1}{2}AB \cdot AC \cdot AD$即得证。
\end{proof}

张角定理的逆也是成立的,即如果平面上四个点满足定理中的等式,就有$B$、$C$、$D$三点共线,这也可以用来处理共线问题。

\subsection{梅涅劳斯(menelaus)定理}
\label{sec:menelaus-theorem}

\begin{theorem}[梅涅劳斯定理]
  设$X$、$Y$、$Z$是三角形$ABC$的三边$AB$、$BC$、$CA$或其延长线上的点,其中有奇数个点在边的延长线上,则此三点共线的充分必要条件是下式成立:
  \begin{equation}
    \label{eq:menelaus-theorem}
    \frac{AX}{XB} \cdot \frac{BY}{YC} \cdot \frac{CZ}{ZA} = 1
 \end{equation}
\end{theorem}
 
\begin{figure}[htbp]
\centering
\includegraphics{content/plane-geometry/pic/menelaus-theorem.pdf}
\caption{梅涅劳斯定理:只有一个点在边的延长线上的情形}
\label{fig:menelaus-theorem}
\end{figure}
 
\begin{figure}[htbp]
\centering
\includegraphics{content/plane-geometry/pic/menelaus-theorem2.pdf}
\caption{梅涅劳斯定理:三个点都在边的延长线上的情形}
\label{fig:menelaus-theorem2}
\end{figure}

只证明必要性,得证之后利用同一法即可轻松得到充分性的证明。
最精巧的是下面这个辅助线的证明
\begin{proof}[证明一]
  过$C$作$XYZ$的平行线,与边$AB$相交于点$P$,那么有
 
\begin{figure}[htbp]
\centering
\includegraphics{content/plane-geometry/pic/menelaus-theorem-proof.pdf}
\caption{梅涅劳斯定理的证明}
\label{fig:menelaus-theorem-proof}
\end{figure}
 
\begin{figure}[htbp]
\centering
\includegraphics{content/plane-geometry/pic/menelaus-theorem-proof2.pdf}
\caption{梅涅劳斯定理的证明}
\label{fig:menelaus-theorem-proof2}
\end{figure}

  \begin{equation*}
    \frac{AX}{XB} \cdot \frac{BY}{YC} \cdot \frac{CZ}{ZA} =
    \frac{AX}{XB} \cdot \frac{BX}{XP} \cdot \frac{PX}{XA} = 1
  \end{equation*}
所以必要性成立,充分性利用同一法即可得证,略去。
\end{proof}

面积方法仍然是行之有效的手段:
\begin{proof}[证明二]
  \begin{equation*}
    \frac{AX}{XB} \cdot \frac{BY}{YC} \cdot \frac{CZ}{ZA} =
    \frac{S_{\triangle AYX}}{S_{\triangle BYX}} \cdot
    \frac{S_{\triangle BYX}}{S_{\triangle CYX}} \cdot
    \frac{S_{\triangle CYX}}{S_{\triangle AYX}} = 1
  \end{equation*}
\end{proof}

再给一个向量方法的证明:
\begin{proof}[证明三]
  记$\vv{AX}=\lambda_1\vv{XB}$, $\vv{BY}=\lambda_2\vv{YC}$, $\vv{CZ}=\lambda_3\vv{ZA}$,则有
  \begin{equation*}
    \vv{XZ} = \vv{AZ}-\vv{AX}=\frac{1}{1+\lambda_3}\vv{AC}-\frac{\lambda_1}{1+\lambda_1}\vv{AB}
  \end{equation*}
  同时
  \begin{eqnarray*}
    \vv{XY} &=& \vv{AY} - \vv{AX} = \left( \frac{1}{1+\lambda_2}\vv{AB}+\frac{\lambda_2}{1+\lambda_2}\vv{AC} \right) - \frac{\lambda_1}{1+\lambda_1}\vv{AB} \\
    &=& \left( \frac{1}{1+\lambda_2}-\frac{\lambda_1}{1+\lambda_1} \right)\vv{AB}+\frac{\lambda_2}{1+\lambda_2}\vv{AC} 
  \end{eqnarray*}
这两个向量共线的充分必要条件是上面这两组系数对应成比例,经过计算即为等式$\lambda_1\lambda_2\lambda_3=-1$,因为是有奇数个点在边的延长线上,所以它等价于定理中的等式。
\end{proof}

从向量形式中我们得到了梅涅劳斯定理的向量表达: $\lambda_1\lambda_2\lambda_3=-1$,而且向量形式的证明,充分性与必要性是统一的。

\subsection{塞瓦(cevian)定理}
\label{sec:cevian-theorem}

\begin{theorem}[塞瓦定理]
  在三角形$ABC$中,点$X$、$Y$、$Z$分别是三边$AB$、$BC$、$CA$上的点,则三条线段$AY$、$BZ$、$CX$交于一点的充分必要条件是
  \begin{equation}
    \label{eq:cevian-theorem}
    \frac{AX}{XB} \cdot \frac{BY}{YC} \cdot \frac{CZ}{ZA} = 1
  \end{equation}
\end{theorem}
 
\begin{figure}[htbp]
\centering
\includegraphics{content/plane-geometry/pic/cevian-theorem.pdf}
\caption{塞瓦定理}
\label{fig:cevian-theorem}
\end{figure}

要注意的是塞瓦定理中三个点都在边上,没有延长线上的点,所以虽然与梅涅劳斯定理中的比例式一样,但却是三线共点,而非三点共线。

%%% Local Variables:
%%% TeX-master: "../../book"
%%% End:
